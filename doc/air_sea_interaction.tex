%Longer version of Air-Sea Interaction 
Through understanding the wave climate in these SWA regions, we gain greater insight into determining at what times during the year remotely and locally forced wind waves dominate the wave field. Wave field's dominated by locally forced waves have strong interactions between waves and the lowest atmospheric layer \cite{cavaleri2012wind} due to the tendency of waves have short frequencies and steep. Processes involved in air-sea interactions that are amplifies by locally forced waves includes wave breaking and whitecapping. Both of these processes leads increase heat and mass fluxes from ejecting sea spray including aerosols into the atmospheric boundary layer, injecting bubbles into the ocean, and causing waving-induced mixing in the upper ocean layer \cite{cavaleri2012wind}. Sea-state dependent surface wave modulated fluxes of momentum, energy, heat and mass are all essential for climate models being able to close budgets to full describe the coupled ocean-atmosphere system \cite{cavaleri2012wind}. Understanding these fluxes begins with knowing the large scale temporally and spatially tendencies of the sea-state of the ocean. Through this study, identification of regions with high tendency for wind-sea dominated wave fields during the spring and summer months are established. From here, we can obtain general expectations for the dominate air-interaction processes present in these regions.

%Short version of Air-Sea Interaction
Sea-state dependent surface wave modulated fluxes of momentum, energy, heat and mass are all essential for climate models \cite{cavaleri2012wind}. Understanding these fluxes begins with knowing the large scale temporally and spatially tendencies of the sea-state of the ocean. Through this study, identification of regions with high tendency for wind-sea dominated wave fields during the spring and summer months are established. From here, we can obtain general expectations for the dominate air-interaction processes present in these regions. 